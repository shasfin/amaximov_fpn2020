\documentclass{article}
%encoding
%--------------------------------------
\usepackage[utf8]{inputenc}
\usepackage[T1]{fontenc}
%--------------------------------------
 
%German-specific commands
%--------------------------------------
\usepackage[ngerman]{babel}
\usepackage{csquotes}
%--------------------------------------

%Pictures
%--------------------------------------
\usepackage{graphicx}
\graphicspath{ {./Pictures/} }
\usepackage{tikz}
\usepackage{subcaption}
\usepackage{float}
\usepackage{wrapfig}
%--------------------------------------

%math
%--------------------------------------
\usepackage{amsmath}
\usepackage{amssymb}
\usepackage{amsfonts}
%--------------------------------------

%Frames
%--------------------------------------
\usepackage{framed}

%Colors
%--------------------------------------
\usepackage{xcolor}
\definecolor{blue-violet}{rgb}{0.54, 0.17, 0.89}
\definecolor{codegreen}{rgb}{0,0.6,0}
\definecolor{codegray}{rgb}{0.5,0.5,0.5}
\definecolor{codepurple}{rgb}{0.58,0,0.82}
\definecolor{backcolour}{rgb}{0.95,0.95,0.92}

%--------------------------------------
\usepackage{multicol}
\usepackage[shortlabels]{enumitem}

%Aufgaben
%--------------------------------------
\usepackage{amsthm}
\newtheorem{aufgabe}{Aufgabe}[section]
\newtheorem{definition}{Definition}[section]
\newtheorem{beispiel}{Beispiel}[section]
\newtheorem*{lernaufgabe*}{Lernaufgabe}
%--------------------------------------

%Listings
%--------------------------------------
\usepackage{ulem}
\usepackage{listings}
 
\lstdefinestyle{mystyle}{
    backgroundcolor=\color{backcolour},   
    commentstyle=\color{codegreen},
    keywordstyle=\color{magenta},
    numberstyle=\tiny\color{codegray},
    stringstyle=\color{codepurple},
    basicstyle=\ttfamily\footnotesize,
    breakatwhitespace=false,         
    breaklines=true,                 
    captionpos=b,                    
    keepspaces=true,                 
    numbers=left,                    
    numbersep=5pt,                  
    showspaces=false,                
    showstringspaces=false,
    showtabs=false,                  
    tabsize=2,
}
 
\lstset{style=mystyle,moredelim=[is][\sout]{|}{|}}
%--------------------------------------


\title{Fliesskommazahlen für Schüler der Gymnasialstufe}
\author{amaximov}
\date{März 2020}

\begin{document}

\maketitle

\tableofcontents

\section{Einführung}
Wie können Computer so viele unterschiedliche Dinge darstellen, wie Text, Bilder und Videos, wenn sie nur Nullen und Einser speichern können?
Wir haben schon gesehen, wie Computer ganze Zahlen darstellen können und letzte Woche hatten wir angefangen zu sehen, wie Computer reelle Zahlen approximieren können. Die Grundidee, wie wir schon gesehen haben, ist im Grunde genommen die Exponentialschreibweise, die wir schon aus der Chemie und aus der Physik kennen. Der Unterschied ist, dass Computer meistens in Basis 2 arbeiten und schränken die Anzahl der möglichen Exponenten und signifikanten Stellen ein.

Um diese Einschränkungen sichtbar und anfassbar zu machen, werden wir folgende Analogie nutzen.
Die Mantisse ist wie ein Wagen. Der Fahrer muss eine Eins sein. Der Wagen hat eine bestimmte Anzahl Sitzplätze. Die Zahlen, die da nicht reinpassen, werden nicht gespeichert. Der Wagen ist mobil und kann vom Universalkomma (alle wissen ja, dass im Zentrum vom Universum das Universalkomma steht, an welchem alle Zahlen dranhängen) zum Speicher gefahren werden. Aber damit man immer weiss, welche Zahl ursprünglich gemeint war, muss man auch wissen, wo der Wagen bezüglich dem Universalkomma stehen soll. Dazu ist ein Nagel zwischen dem Fahrersitz und dem ersten Passagiersitz eingehämmert und daran ist ein Seil gebunden. Dieser Seil hat zwei Enden: das negative Ende (nach links) und das positive Ende (nach rechts). Am Seil wird markiert, wie weit das Universalkomma ist.

\begin{beispiel}
Wir werden jetzt zusammen die Zahl 6.5 darstellen.

Am Universalkomma hängt die Zahl dran.
Der Wagen mit 5 Plätze kommt. Der Fahrer ist die grösste Zahl. Das ist ja das wichtigste. Wenn wir nichts anderes von der Zahl wissen werden als das, was im Wagen steht, wollen wir die signifikanteste Stelle sicher im Wagen drin haben.

[Bild mit Wagen]

Wir verbinden das Universalkomma mit dem Nagel am Wagen und markieren die Stelle am Seil.

Jetzt können wir die Zahl speichern

[Bild im Speicher]

Im Computer wird das als Bitmuster gespeichert. Zuerst Vorzeichen, dann Exponent (Markierungen am Seil), dann Mantisse (die erste Eins ist eigentlich überflüssig).

Dank dem Seil kann man den Wagen wiederherstellen im Bezug auf das Universalkomma und wieder ausrechnen, was das für eine Zahl war.

\end{beispiel}

\begin{beispiel}
Nicht alle Zahlen lassen sich im Fliesskommazahlensystem genau darstellen. Manche müssen gerundet werden.
Zum Beispiel, die Zahl 10.75 würde in Binär so aussehen (hängt am Universalkomma).

[Bild mit Universalkomma]

Dann kommt der Wagen mit 5 Plätze. Die erste Eins und noch 4 Bits steigen ein. Das Seil wird entsprechend gebunden und markiert.

[Bild mit Wagen]

Die letzte Eins wird nicht gespeichert. Es gibt kein Platz. Sie geht verloren.
\end{beispiel}
\newpage

\section{Fliesskommazahlen}
\subsection{Kleinste und grösste positive Zahlen}

Die Länge vom Seil (die Anzahl der möglichen Exponenten) und die Sitzplätze im Wagen (die verfügbaren Bits in der Mantisse) sind endlich. Deswegen können wir nur eine endliche Anzahl von Zahlen darstellen. Dann gibt es unter den endlich vielen positiven Zahlen eine grösste und eine kleinste.

\begin{beispiel}
Mantissenlänge: 5 Bits, Exponentenbereich: -3 bis 3.

Wie konstruieren wir die grösste Zahl?

Was wäre die Mantisse? So gross wie möglich. 1.1111

Was wäre der Exponent? So gross wie möglich. 3

Wagen-Darstellung:

Bitmuster: 0 111 (1)1111

Mathematische Darstellung: \(1.1111 \cdot 2^3\)

Wert: 15.5

\end{beispiel}

\begin{beispiel}
Mantissenlänge: 5 Bits, Exponentenbereich: -3 bis 3.

Wie konstruieren wir die kleinste Zahl?

Was wäre die Mantisse? So klein wie möglich, dennoch mit einer führenden Eins: 1.0000

Was wäre der Exponent? So klein wie möglich. -3

Wagen-Darstellung:

Bitmuster: 0 001 (1)0000

Mathematische Darstellung: \(1.0000 \cdot 2^{-3}\)

Wert: 1/8
\end{beispiel}

\begin{aufgabe}
Mantissenlänge: 3 Bits, Exponentenbereich: -1 bis 1.
Finde die grösste und kleinste positive Zahlen in diesem Fliesskommasystem. Stelle sie als Bitmuster und in der Exponentialschreibweise dar (Wert im Dezimalsystem angeben).
(Für den Exponentenbereich: 01 kodiert -1, 10 kodiert 0, 11 kodiert 1).
\end{aufgabe}

\begin{aufgabe}
Mantissenlänge: 4 Bits, Exponentenbereich: -1 bis 1, kodiert wie in der vorherigen Aufgabe.
Finde die grösste und kleinste positive Zahlen (Werte im Dezimalsystem angeben). Stelle sie als Bitmuster und in der Exponentialschreibweise dar.
\end{aufgabe}

Im Allgemeinen für einen Fliesskommasystem, wo die Mantisse m Bits lang ist und der Exponentenbereich von \(e_min\) bis \(e_max\) geht, findet man die grösste positive Zahl indem man in der Mantisse nur Einser schreibt, so dass diese so gross wie möglich wird: \(1.11111 \ldots 111\) und den Exponenten so gross wie möglich wählt: \(e_max\). Die grösste Zahl ist also:
\[1.1111111 \ldots 111 \cdot 2^{e_{max}}\]
Und hat den Bitmuster: 0 1111...11 (1)11111...11.

Die kleinste positive Zahl findet man, indem man die Mantisse so klein wie möglich wählt: \(1.00000 \ldots 000\) und den Exponenten so klein wie möglich: \(e_min\). Die kleinste Zahl ist also:
\[1.0000000 \ldots 000 \cdot 2^{e_{min}}\]
Und hat den Bitmuster: 0 0000...01 (1)000000...0.

\subsection{Darstellbare Zahlen}

In diesem Abschnitt werden wir ein Gefühl dafür bekommen, welche Zahlen sie darstellen lassen und welche nicht.
Hier und in den folgenden Kapiteln, falls nicht speziell vermerkt, werden wir mit Mantissenlänge 5 und Exponentenbereich von -3 bis 3 arbeiten.

\begin{beispiel}
Nehmen wir eine Zahl zwischen 1/8 und 15.5 (die kleinste und grösste darstellbare Zahlen in diesem Fliesskommasystem), zum Beispiel 10.25. Lässt sich diese Zahl darstellen?

[Bild mit Wagen]

Diese Zahl lässt sich in gegebenem System nicht exakt darstellen, weil in der Mantisse gibt es Platz nur für 5 Bits, so geht die letzte 1 verloren.
\end{beispiel}

\begin{beispiel}
Wir werden jetzt die nächste und die vorherige darstellbare Zahl von 1 finden.

[Bild mit Wagen]
\begin{figure}[H]
\centering
\includegraphics[width=\linewidth]{Pictures/Nachbarn.png} 
\end{figure}

Die nächste Zahl finden wir, indem wir in der Mantisse ganz rechts eine Eins setzen. Das ist die nächstgrösste Mantisse nach 1.0000. Der Exponent bleibt gleich. Die nächste Zahl ist also 1 + 1/16.

Die grösste Zahl kleiner als Eins finden wir, indem wir die Mantisse so gross wie möglich machen und den Exponenten um eins verkleinern. Die vorherige Zahl ist also 1 - 1/32.
\end{beispiel}

\begin{aufgabe}
Finde den nächstkleinsten und nächstgrössten darstellbaren Nachbar von den folgenden Zahlen. Schreibe die Werte im Dezimalsystem auf und stelle sie als Bitmuster und in der Exponentialschreibweise dar. Sind alle Nachbarn gleich entfernt?
\begin{enumerate}[(a)]
\item 2
\item 3
\item 4
\end{enumerate}
\end{aufgabe}

\subsubsection*{\textcolor{blue-violet}{Teste dich selber}}

\begin{aufgabe}\label{fliesskommazahlen_kontrollfragen}
Beantworte folgende Fragen:
\begin{enumerate}[(a)]
\item Kann man im Fliesskommazahlensystem alle reelle Zahlen darstellen?
\item Gibt es eine grösste Zahl im Fliesskommazahlensystem? Falls nein, warum? Falls ja, wie findet man sie?
\item Gibt es eine kleinste Zahl im Fliesskommazahlensystem? Falls nein, warum? Falls ja, wie findet man sie?
\item Warum kann man nicht alle Zahlen zwischen f\_min und f\_max darstellen? Gib ein Beispiel von einer Zahl, die sich nicht im System mit Mantissenlänge 3 und Exponentenbereich von -1 bis 1 darstellen lässt.
\item Sind alle Zahlen im Fliesskommazahlensystem gleichverteilt oder gibt es kleinere und grössere Löcher? Falls es Löcher gibt, wie sind sie verteilt? Wo sind sie grösser, wo sind sie kleiner?
\end{enumerate}
\end{aufgabe}

\newpage

\section{Addition}
Im vorherigen Kapitel haben wir gesehen, welche Zahlen in einem Fliesskommazahlensystem dargestellt werden können, das heisst welche Zahlen  exakt in einem Computer gespeichert werden können. Computer werden aber nicht nur zum Speichern von Zahlen verwendet, sondern auch für Berechnungen.  Auch bei Berechnungen verhalten sich Fliesskommazahlen nicht ganz wie reelle Zahlen. In diesem Kapitel werden wir dies am Beispiel der Addition erfahren. 

\begin{beispiel}
Wir möchten \(1/4 + 1/8\) ausrechnen. Der erste Schritt ist beide Zahlen aufzuschreiben. Wie in den vorherigen Kapiteln, sind in violett die reelle Zahlen in Basis 2 aufgeschrieben und braune ''Kasten'' mit orangenem ''Seil'' verwendet, um Mantisse und Exponent zu veranschaulichen. Rechts wird das Bitmuster in der gewöhnlichen Form angegeben.
\begin{figure}[H]
\centering
\includegraphics[width=\linewidth]{Pictures/Addition1-4and1-8_1.png}
\end{figure}

Damit wir die Bits der Mantisse stellenweise addieren können, wie wir das von den ganzen Zahlen kennen, müssen wir die zwei ''Kasten'' so verschieben, dass sie sich untereinander befinden. Da alles, was ausserhalb vom ''Kasten'' landet, verloren geht, werden wir den Kasten von der kleineren Zahl unter den Kasten von der grösseren Zahl schieben. So werden wir die Stellen mit dem niedrigsten Wert verlieren. In diesem Fall verlieren wir eine Null, der Wert der Zahl verändert sich also nicht.

Beachte, dass wenn der Kasten verschoben wird, verschiebt sich auch die Markierung bezüglich des ''Seils'', das heisst der Exponent verändert sich. Die Markierung am ''Seil'' bleibt immer unter dem Komma.
\begin{figure}[H]
\centering
\includegraphics[width=\linewidth]{Pictures/Addition1-4and1-8_2.png}
\end{figure}

Wenn die Kasten untereinander sind, können wir die Bits in den Kasten wie gewöhnlich addieren, wie bei den Integers.
\begin{figure}[H]
\centering
\includegraphics[width=\linewidth]{Pictures/Addition1-4and1-8_3.png}
\end{figure}
Wir haben ausgerechnet, dass \(1/4 + 1/8 = 3/8\), in der Exponentialschreibweise \(1.1000 \cdot 2^{-2}\).
\end{beispiel}

\begin{beispiel}
Wir möchten \(2+3\) ausrechnen. Im ersten Schritt schreiben wir beide Zahlen auf.
\begin{figure}[H]
\centering
\includegraphics[width=\linewidth]{Pictures/Addition2and3_1.png}
\end{figure}
Die Kasten befinden sich schon untereinander. Wir müssen also nichts verschieben und können sofort losrechnen.
\begin{figure}[H]
\centering
\includegraphics[width=\linewidth]{Pictures/Addition2and3_2.png}
\end{figure}
Beachte, dass der Kasten vom Ergebnis bezüglich den Kasten der Summanden verschoben ist, um die neue signifikante Stelle zu enthalten.

Wir haben ausgerechnet, dass \(2 + 3 = 5\), in der Exponentialschreibweise \(1.0100 \cdot 2^2\).
\end{beispiel}

\begin{aufgabe}\label{addition}
Rechne folgende Summen aus. Die Mantissenlänge beträgt \(5\) Bits, der Exponent geht von \(-3\) bis \(3\). Gebe bitte das Bitmuster und die Exponentialdarstellung des Resultats an.
\begin{enumerate}[(a)]
\item \(5/8 + 3/4\)
\item \(10 + 2.25\)
\item \(17/16 + 2\)
\end{enumerate}
\end{aufgabe}

Die Addition im Fliesskommazahlensystem ist wie gewöhnlich kommutativ, weil wir immer die kleinste Zahl so verschieben, dass ihr Kasten unter dem Kasten der grösseren Zahl steht und dann die Bits in beiden Kasten stellenweise zusammen addieren. In der folgenden Aufgabe werden wir prüfen, ob sie auch assoziativ ist.

\begin{lernaufgabe*}\label{lernaufgabe_assoziativ}
Du wirst jetzt herausfinden, ob die Addition bei den Fliesskommazahlen assoziativ ist. Berechne dazu zwei Mal die gleiche Summe in einem Fliesskommazahlensystem mit Mantissenlänge \(5\) und Exponenten von \(-3\) bis \(3\):
Das erste Mal als \(1/8 + 2/8 + 3/8 + 4/8 + 5/8 + 6/8 + 7/8 + 8/8\) und das zweite Mal als \(8/8 + 7/8 + 6/8 + 5/8 + 4/8 + 3/8 + 2/8 + 1/8\).

Welchen Resultat erwartest du? Sind die zwei Summen gleich oder unterschiedlich? Kannst du daraus folgern, ob Addition assoziativ ist?
Nimm dir Zeit und rechne die zwei Summen tatsächlich aus.
\end{lernaufgabe*}

Die zwei Summen, die du ausgerechnet hast, liefern unterschiedliche Ergebnisse. Die erste liefert den exakten Wert \(4.5\), während bei der zweiten Summe kriegen wir im Fliesskommazahlensystem nur \(4.25\), und das obwohl der exakte Wert dargestellt werden kann. Das passiert, weil man bei den Fliesskommazahlen nur Zahlen der ähnlichen Grössenordnung exakt addieren kann. In der ersten Summe addieren wir die kleineren Summanden am Anfang, wenn die kumulative Summe noch nicht zu gross ist. In der zweiten Summe wächst die kumulative Summe sehr schnell, und irgendwann sind die Summanden zu klein bezüglich der kumulativen Summe, um einen Unterschied zu machen.

Daraus können wir folgern, dass die Addition bei den Fliesskommazahlen nicht assoziativ ist.


\begin{aufgabe}\label{ein_achtel}
Betrachten wir die Summe \(1/8 + 1/8 + 1/8 + \dotsb + 1/8\).
Bei den reellen Zahlen können wir mit solchen Summen auf beliebig grossen Zahlen kommen. Bei den Fliesskommazahlen kann das nicht gehen, weil, wie wir im vorherigen Kapitel gesehen haben, es eine grösste Fliesskommazahl gibt. Aber können wir diese Zahl auch tatsächlich erreichen?

In einem Fliesskommazahlensystem mit Mantissenlänge \(5\) und Exponentenbereich von \(-3\) bis \(3\), was ist die grösste Zahl, die wir erreichen können, wenn wir beliebig viele \(1/8\) zusammen rechnen? Wie viele Summanden brauchen wir, um diese Zahl zu erreichen?
\end{aufgabe}


\subsubsection*{\textcolor{blue-violet}{Teste dich selber}}
\begin{aufgabe}\label{addition_kontrollfragen}
Beantworte folgende Fragen:
\begin{enumerate}[(a)]
\item Warum kann man im Allgemeinen zwei Mantissen nicht stellenweise zusammen addieren?
\item Gregory behauptet, dass der Kasten vom Ergebnis sich immer genau unter dem Kasten der grössten Zahl befindet. Hat er recht? Argumentiere.
\item Hannah behauptet, dass die Addition bei den Fliesskommazahlen nicht kommutativ und nicht assoziativ ist. Hat sie recht? Argumentiere.
\end{enumerate}
\end{aufgabe}

\begin{aufgabe}\label{ameisenkönigin}
Die Ameisenkönigin möchte ausrechnen, wie viele Ameisen braucht sie, um \(10\) Reiskörnchen zu transportieren. Sie weiss, dass eine Ameise allein \(1/4\) Reiskorn transportiert. Die Ameisenkönigin hat dazu folgendes Programm geschrieben.
\begin{lstlisting}[language=Python, caption={Programm von der Ameisenkönigin}]
def nof_ameisen():
    sum = 0.0
    i = 0
    while node != 10.0:
    	i += 1
       sum += 0.25
    return i
\end{lstlisting}
Die Ameisencomputer arbeiten mit Fliesskommazahlen mit Mantissenlänge \(5\) und Exponenten zwischen \(-3\) und \(3\).
Kann die Ameisenkönigin mit diesem Programm die gewünschte Anzahl Ameisen herausfinden? Falls ja, wie viele Ameisen braucht sie, um 10 Reiskörnchen zu transportieren laut diesem Programm? Falls nein, was ist die maximale Summe, die das Programm erreichen kann?
\end{aufgabe}
\newpage

\section{Zusammenfassung}
Wir haben gesehen, wie man im Computer reelle Zahlen durch \textbf{Fliesskommazahlen} approximieren kann. Da wir eine endliche Darstellung verwenden, gibt es eine endliche Anzahl Zahlen, die wir darstellen können. Es gibt also eine grösste und eine kleinste positive Zahl.

Die \textbf{grösste positive Zahl} kriegen wir, wenn wir die grösstmögliche Mantisse mit dem grössmöglichen Exponenten kombiniren, d.h. die Mantisse besteht aus lauter Einser und der Exponent ist maximal.

Die \textbf{kleinste positive Zahl} kriegen wir, wenn wir die kleinstmögliche Mantisse mit dem kleinstmöglichen Exponenten kombinieren. Da wir in der Darstellung verlangen, dass das erste Bit der Mantisse eine Eins ist, besteht die kleinstmögliche Mantisse aus einer führenden Eins und vielen Nullen. Der Exponent ist minimal.

Wir haben gelernt, die nächste und die vorherige darstellbare Zahl zu bestimmen. So haben wir gesehen, dass darstellbare Zahlen nicht gleichverteilt auf der Zahlengerade auftreten, sondern dass der \textbf{Abstand} zwischen benachbarten darstellbaren Zahlen wächst, wenn die Zahlen grösser werden.

Mit den Fliesskommazahlen kann man auch rechnen. Wir haben insbesondere die \textbf{Addition} kennengelernt.

Wenn man zwei Fliesskommazahlen zusammen addieren möchte, muss man sie zuerst zum gleichen Exponenten bringen, dann kann man die neuen Mantissen wie ganze Zahlen addieren. Am Schluss muss man sicherstellen, dass der Exponent und die Mantisse gültig sind: der Exponent muss zwischen dem minimalen und dem maximalen Exponenten sein, die Mantisse muss eine führende Eins haben.

Wir haben gesehen, dass die Addition im Fliesskommazahlensystem Unterschiede zur Addition bei den reellen Zahlen aufweist. Erstens, nicht alle darstellbare Zahlen lassen sich exakt addieren. Zweitens, die Addition bei den Fliesskommazahlen ist \textbf{nicht assoziativ}. Es kann einen Unterschied machen, welche Teilsummen man zuerst berechnet.

Die Fliesskommazahlen sind eine mächtige Darstellung, die mit wenig Bits sehr unterschiedliche Zahlen speichern kann. Das hat aber auch seine Grenzen. Wir müssen in Kauf nehmen, dass die Resultate der Berechnungen nicht immer exakt sind.

\newpage

\nocite{*}
\bibliographystyle{plain}
\bibliography{refs}
\newpage

\section{Beispiellösungen}
\subsection{Einführung}
\paragraph{Aufgabe \ref{einleitung-vom-kasten-nach-zahl}}
Das Vorzeichen ist positiv. Die Mantisse übernehmen wir aus dem Kasten. Die Kodierung vom Exponenten können wir am Seil ablesen. Den Exponenten bestimmen wir, indem wir die \texttt{100} auf dem Seil als \(0\) interpretieren und die Stellen zwischen der Null und der Markierung zählen. In diesem Fall sind es \(-2\) Stellen.
\begin{figure}[H]
\centering
\includegraphics[width=0.8\linewidth]{Pictures/Einleitung_from_Kasten_Loesung.png} 
Den Dezimalwert berechenen wir, indem wir \((-1)^0 \cdot 1.001 \cdot 2^{-2} = 0.01001\) nach Dezimal konvertieren. In diesem Fall erhalten wir \(1/4 + 1/32 = 9/32\).
\end{figure}

\subsection{Fliesskommazahlen}

\paragraph{Aufgabe \ref{kleinsteZahl-5-3}}
Wir konstruieren die kleinste positive Zahl im Fliesskommazahlensystem mit Mantissenlänge \(5\) und Exponenten von \(-3\) bis \(3\).

Als erstes platzieren wir den Kasten. Damit die Zahl möglichst klein wird, muss der Kasten nach rechts möglichst weit weg vom Komma stehen. Wir haben aber eine Einschränkung: Das Seil muss immer mit dem Komma verbunden bleiben.
\begin{figure}[H]
\centering
\includegraphics[width=0.85\linewidth]{Pictures/kleinsteZahl1.png}
\end{figure}
Der Exponent muss also möglichst klein sein.

Was ist mit der Mantisse? Sicher muss eine Eins an der ersten Stelle stehen.
\begin{figure}[H]
\centering
\includegraphics[width=0.85\linewidth]{Pictures/kleinsteZahl2.png}
\end{figure}

Damit die Mantisse möglichst klein wird, müssen wir so viele Stellen wie möglich auf Null setzen.
\begin{figure}[H]
\centering
\includegraphics[width=0.85\linewidth]{Pictures/kleinsteZahl3.png}
\end{figure}

Die kleinste darstellbare Zahl in diesem Fliesskommazahlensystem ist also \(1/8\).

\paragraph{Aufgabe \ref{groesste-kleinste-4-3}}
Die grösste positive darstellbare Zahl in einem Fliesskommazahlensystem mit Mantissenlänge \(4\) und Exponent zwischen \(-3\) und \(3\) ist \(15\). Das ist nicht viel kleiner als \(15.5\), die grösste positive darstellbare Zahl in einem Fliesskommazahlensystem mit einem Bit mehr für die Mantisse.
\begin{figure}[H]
\centering
\includegraphics[width=0.85\linewidth]{Pictures/groessteZahl-4-3.png}
\end{figure}

Die kleinste positive darstellbare Zahl in einem Fliesskommazahlensystem mit Mantissenlänge \(4\) und Exponent zwischen \(-3\) und \(3\) ist auch \(1/8\), genau wie die kleinste positive darstellbare Zahl in einem Fliesskommazahlensystem mit einem Bit mehr für die Mantisse.
\begin{figure}[H]
\centering
\includegraphics[width=0.85\linewidth]{Pictures/kleinsteZahl-4-3.png}
\end{figure}

Wie wir sehen, die Länge der Mantisse scheint wenig Einfluss auf die grösste und kleinste positive darstellbare Zahlen zu haben.

\paragraph{Aufgabe \ref{groesste-kleinste-5-2}}
\begin{enumerate}[(a)]
\item Der Exponent liegt zwischen \(-1\) und \(1\) und die mögliche Kodierungen sind \texttt{01, 10, 11}.
\item Die Erwartung ist, dass die grösste positive darstellbare Zahl deutlich kleiner wird, weil das Seil viel kürzer ist, und wir den Kasten nicht mehr so weit nach links ziehen können, wie im Fliesskommazahlensystem mit \(3\) Bits für den Exponenten. Analog, die kleinste positive darstellbare Zahl wird deutlich grösser.
\item Die grösste positive darstellbare Zahl in diesem System ist \(3+7/8 = 31/8\). Wie erwartet, das ist viel grösser als in einem Fliesskommazahlensystem mit einem Bit mehr für den Exponenten.
\begin{figure}[H]
\centering
\includegraphics[width=0.85\linewidth]{Pictures/groessteZahl-5-2.png}
\end{figure}

Die kleinste positive darstellbare Zahl in diesem System ist \(0.5\). Das ist viel grösser als in einem Fliesskommazahlensystem mit einem Bit mehr für den Exponenten.
\begin{figure}[H]
\centering
\includegraphics[width=0.85\linewidth]{Pictures/kleinsteZahl-5-2.png}
\end{figure}

\end{enumerate}


\paragraph{Aufgabe \ref{groesste-kleinste-allgemein}}

Im Allgemeinen für einen Fliesskommazahlensystem mit Mantissenlänge \(m\) und Exponenten zwischen \(e_{min}\) und \(e_{max}\) findet man die grösste und kleinste positive Zahlen wie folgt.

Für die grösste positive Zahl wählt man den grösstmöglichen Exponenten \(e_{max}\) und die grösstmögliche Mantisse \(1.111 \ldots 111\). In der Exponentialschreibweise ist die grösste Zahl also 
\[1.1111111 \ldots 111 \cdot 2^{e_{max}}\]
und hat das Bitmuster \texttt{0 1111...111 (1)111111...111}.

Für die kleinste positive Zahl wählt man den kleinsten möglichen Exponenten \(e_{min}\) und die kleinste mögliche Mantisse. Beachte, dass die Mantisse immer mit einer Eins starten muss. Die kleinste mögliche Mantisse ist deswegen \(1.0000 \ldots 000\). In der Exonentialschreibweise ist die kleinste Zahl also
\[1.0000000 \ldots 000 \cdot 2^{e_{min}}\]
und hat das Bitmuster \texttt{0 0000...001 (1)00000000...000}.
%--------------------------------

\paragraph{Aufgabe \ref{nachbarn-vorherige}}
Die nächstkleinste, oder vorherige, darstellbare Zahl finden wir, indem wir die Mantisse kleiner zu machen versuchen. Da die Mantisse von \(1\) die kleinste mögliche Mantisse ist, müssen wir den Exponenten um Eins zurücksetzen und die grösstmögliche Mantisse wählen.

\begin{figure}[H]
\centering
\includegraphics[width=\linewidth]{Pictures/Nachbarn1_P.png} 
\end{figure}
Die vorherige Zahl ist also \(31/32\).

Beachte, dass der Abstand zur nächsten und vorherigen darstellbaren Zahlen in diesem Fall nicht symmetrisch ist: die nächste Zahl ist \(1/16\) entfernt, während die vorherige nur \(1/32\).

\paragraph{Aufgabe \ref{nachbarn}}

\begin{enumerate}[(a)]
\item Die Nachbarn von \(2\) sind \(31/16\) und \(17/8\).
\begin{figure}[H]
\centering
\includegraphics[width=\linewidth]{Pictures/Nachbarn2.png}
\end{figure}

\item Die Nachbarn von \(3\) sind \(23/8\) und \(25/8\).
\begin{figure}[H]
\centering
\includegraphics[width=\linewidth]{Pictures/Nachbarn3.png}
\end{figure}

\item Die Nachbarn von \(4\) sind \(31/8\) und \(17/4\).
\begin{figure}[H]
\centering
\includegraphics[width=\linewidth]{Pictures/Nachbarn4.png}

Die positive darstellbare Zahlen sind auf dem Zahlenstrahl nicht gleichverteilt.
\end{figure}

\end{enumerate}

\paragraph{Aufgabe \ref{nachbarn-laenge}}
\begin{enumerate}[(a)]
\item Im Fliesskommazahlensystem mit Mantissenlänge \(4\) und Exponent zwischen \(-3\) und \(3\) sind die Nachbarn von \(1\): \(15/16\) und \(9/8\).
\begin{figure}[H]
\centering
\includegraphics[width=\linewidth]{Pictures/Nachbarn1-4-3-Loesung.png}
\end{figure}

\item Im Fliesskommazahlensystem mit Mantissenlänge \(5\) und Exponent zwischen \(-1\) und \(1\) sind die Nachbarn von \(1\): \(31/32\) und \(17/16\).
\begin{figure}[H]
\centering
\includegraphics[width=\linewidth]{Pictures/Nachbarn1-5-2-Loesung.png}
\end{figure}

\item Die Länge der Mantisse beeinflusst den Abstand zwischen darstellbaren Zahlen stärker als die Länge der Exponentenkodierung. Wenn der Kasten grösser ist, gibt es mehr Platz für signifikante Stellen und Zahlen können genauer approximiert werden. Das führt dazu, dass der Abstand zwischen darstellbaren Zahlen kleiner wird.
\end{enumerate}

%--------------------------------

\paragraph{Aufgabe \ref{fliesskommazahlen_kontrollfragen}}
\begin{enumerate}[(a)]
\item Nein, es gibt unendlich viele reelle Zahlen und endlich viele Fliesskommazahlen.
\item Ja, die grösste Zahl ist \(1.1111 \ldots 111 \cdot 2^{e_{max}}\).
\item Ja, die kleinste Zahl ist \(1.0000 \ldots 000 \cdot 2^{e_{min}}\).
\item Die Länge der Exponentenkodierung beeinflusst den Bereich stärker als die Mantissenlänge. Wenn das Seil länger ist, kann man den Kasten weiter weg vom Komma platzieren und viel grössere oder kleinere Zahlen darstellen.
\item Zum Beispiel, die Zahl \(2.25\) lässt sich in diesem System nicht exakt darstellen. In der Exponentialschreibweise diese Zahl ist \(1.001 \cdot 2^{1}\). Um diese Zahl exakt zu speichern bräuchten wir \(4\) Bits für die Mantisse, wir haben aber nur \(3\).
\item Nein, die darstellbare Fliesskommazahlen sind nicht gleichverteilt. Die kleineren stehen dichter beieinander, weil bei kleineren Zahlen die letzte Stelle der Mantisse weniger Wert ist.
\item Die Mantissenlänge beeinflusst stärker den Abstand zwischen positiven darstellbaren Zahlen in einem Fliesskommazahlensystem. Wenn der Kasten mehr Plätze hat, kann man mehr Stellen speichern und somit Zahlen genauer darstellen.
\end{enumerate}

%--------------------------------

\subsection{Addition}
\paragraph{Aufgabe \ref{addition}}
\begin{enumerate}[(a)]
\item \(5/8 + 3/4 = 11/8\), in der Exponentialschreibweise \(1.0110 \cdot 2^{0}\)

Im ersten Schritt schreiben wir die Zahlen auf.
\begin{figure}[H]
\centering
\includegraphics[width=\linewidth]{Pictures/Addition5-8and3-4_1.png}
\end{figure}
Da die zwei Kasten schon übereinander liegen, müssen wir sie nicht verschieben und können die Bits stellenweise zusammen addieren.
\begin{figure}[H]
\centering
\includegraphics[width=\linewidth]{Pictures/Addition5-8and3-4_2.png}
\end{figure}
Der Kasten vom Ergebnis ist verschoben bezüglich den Kasten der Summanden.

\item \(10 + 2.25 = 12\), in der Exponentialschreibweise \(1.1000 \cdot 2^3\)

Im ersten Schritt schreiben wir die Zahlen auf.
\begin{figure}[H]
\centering
\includegraphics[width=\linewidth]{Pictures/Addition10and2-25_1.png}
\end{figure}

Im zweiten Schritt schiben wir den Kasten von der kleinsten Zahl unter den Kasten der grössten Zahl. Dabei gehen zwei Stellen verloren, eine davon ist eine Eins.
\begin{figure}[H]
\centering
\includegraphics[width=\linewidth]{Pictures/Addition10and2-25_2.png}
\end{figure}

Nun können wir die Bits stellenweise zusammenrechnen.
\begin{figure}[H]
\centering
\includegraphics[width=\linewidth]{Pictures/Addition10and2-25_3.png}
\end{figure}

\item \(17/16 + 2 = 3\), in der Exponentialschreibweise \(1.1000 \cdot 2^1\).

Im ersten Schritt schreiben wir die Zahlen auf.
\begin{figure}[H]
\centering
\includegraphics[width=\linewidth]{Pictures/Addition17-16and2_1.png}
\end{figure}

Im zweiten Schritt schiben wir den Kasten von der kleinsten Zahl unter den Kasten der grössten Zahl. Dabei geht eine Stelle verloren.
\begin{figure}[H]
\centering
\includegraphics[width=\linewidth]{Pictures/Addition17-16and2_2.png}
\end{figure}

Nun können wir die Bits stellenweise zusammenrechnen.
\begin{figure}[H]
\centering
\includegraphics[width=\linewidth]{Pictures/Addition17-16and2_3.png}
\end{figure}

\end{enumerate}

%--------------------------------

\paragraph{Aufgabe \ref{ein_achtel}} Die maximale Zahl, die wir erreichen können, wenn wir \(1/8 + 1/8 + \dotsb + 1/8\) zusammen rechnen, ist \(4.0\).

Zum einen, wenn wir die \(4.0\) erreicht haben, kommen wir nicht mehr weiter.
Das sehen wir, wenn wir \(4.0 + 1/8\) ausrechnen. Wie gewöhnlich schreiben wir zuerst die Summanden untereinander.
\begin{figure}[H]
\centering
\includegraphics[width=\linewidth]{Pictures/Addition4and1-8_1.png}
\end{figure}
Wenn wir den Kasten von \(1/8\) unter den Kasten von \(4.0\) verschieben, sehen wir, dass alle signifikanten Stellen von \(1/8\) verloren gehen, auch die führende Eins.
\begin{figure}[H]
\centering
\includegraphics[width=\linewidth]{Pictures/Addition4and1-8_2.png}
\end{figure}
Deswegen, wenn wir \(4.0 + 1/8\) ausrechnen, kriegen wir \(4.0\).
\begin{figure}[H] 
\centering
\includegraphics[width=\linewidth]{Pictures/Addition4and1-8_3.png} 
\end{figure}
Egal wie viele \(1/8\) rechnen wir zusammen, bleiben wir immer bei \(4.0\).

Jetzt bleibt uns zu zeigen, dass wir die \(4.0\) auch tatsächlich erreichen können. Das Problem bei der \(4.0\) ist, dass alle signifikanten Stellen von \(1/8\) verloren gehen. Das passiert, weil der Unterschied zwischen dem Exponenten von \(4.0\) und dem Exponenten von \(1/8\) die ganze Mantissenlänge beträgt. Das passiert bei einem kleineren Exponenten nicht. Zum Beispiel, wenn wir \(2.0 + 1/8\) ausrechnen, sehen wir, dass das Ergebnis wie erwartet \(17/8\) ist.

Um zu zeigen, dass das Problem erst bei \(4.0 + 1/8\) auftritt, rechnen wir \(2.0 + 1/8\). Das Ergebnis ist wie erwartet \(17/8\).
\begin{figure}[H]
\centering
\includegraphics[width=\linewidth]{Pictures/Addition2and1-8_1.png} 
\includegraphics[width=\linewidth]{Pictures/Addition2and1-8_2.png} 
\includegraphics[width=\linewidth]{Pictures/Addition2and1-8_3.png} 
\end{figure}
Wir erreichen also die \(4.0\) nach 32 Summanden und kommen dann nicht mehr weiter.

%--------------------------------

\paragraph{Aufgabe \ref{addition_kontrollfragen}}
\begin{enumerate}[(a)]
\item Der Wert der Bits in der Mantisse hängt vom Exponenten ab. Zum Beispiel, dieselbe Mantisse \(1.0000\) mit unterschiedlichen Exponenten kann \(4\), \(2\), \(1\), \(1/2\), \(1/4\) und \(1/8\) darstellen. Wir wollen nicht, dass \(1+2\) das gleiche Ergebnis liefert die \(1 + 1/4\). Wir wollen nur Bits mit dem gleichen Wert zusammen addieren. Deswegen müssen wir vor der Addition sicherstellen, dass die Kasten der beiden Summanden exakt untereinander stehen.

\item Die Aussage von Gregory ist falsch. Der Kasten vom Ergebnis kann sich bewegen bezüglich des Kastens vom grössten Summanden. Dies passiert, zum Beispiel, wenn man \(2.5 + 1.75\) ausrechnet.

\item Hannah hat teilweise recht. Die Addition bei den Fliesskommazahlen ist kommutativ aber nicht assoziativ.

Wenn wir zwei Zahlen zusammen addieren und diese zwei Zahlen vertauschen, kriegen wir das gleiche Ergebnis auch bei Fliesskommazahlen.

Wenn wir aber die Reihenfolge verändern, in welcher die Zahlen zusammengerechnet werden, können wir unterschiedlich Ergebnisse bekommen. Das passiert, weil wir nur dann den exakten Wert ausrechnen können, wenn die Grössenordnung der Teilsummanden ähnlich ist.
\end{enumerate}

%--------------------------------

\paragraph{Aufgabe \ref{ameisenkönigin}}

Nein, das Programm der Ameisenkönigin wird unendlich lange laufen und die Anzahl Ameisen, die es braucht, um 10 Reiskörnchen zu transportieren, nie ausgeben. Das Problem ist analog zu dem, was wir in Aufgabe \ref{ein_achtel} gesehen haben. Das Programm läuft wie erwartet bis wir die \(8.0\) erreichen. Wenn wir aber \(1/4\) dazu rechnen, dann verlieren wir alle signifikanten Stellen von \(1/4\) und die \(8.0\) bleibt unverändert.
\begin{figure}[H]
\centering
\includegraphics[width=\linewidth]{Pictures/Addition8and1-4_1.png} 
\includegraphics[width=\linewidth]{Pictures/Addition8and1-4_2.png} 
\includegraphics[width=\linewidth]{Pictures/Addition8and1-4_3.png} 
\end{figure}

\end{document}
