Wir haben gesehen, wie man im Computer reelle Zahlen durch \textbf{Fliesskommazahlen} approximieren kann. Da wir eine endliche Darstellung verwenden, gibt es eine endliche Anzahl Zahlen, die wir darstellen können. Es gibt also eine grösste und eine kleinste positive Zahl.

Die \textbf{grösste positive Zahl} kriegen wir, wenn wir die grösstmögliche Mantisse mit dem grössmöglichen Exponenten kombiniren, d.h. die Mantisse besteht aus lauter Einser und der Exponent ist maximal.

Die \textbf{kleinste positive Zahl} kriegen wir, wenn wir die kleinstmögliche Mantisse mit dem kleinstmöglichen Exponenten kombinieren. Da wir in der Darstellung verlangen, dass das erste Bit der Mantisse eine Eins ist, besteht die kleinstmögliche Mantisse aus einer führenden Eins und vielen Nullen. Der Exponent ist minimal.

Wir haben gelernt, die nächste und die vorherige darstellbare Zahl zu bestimmen. So haben wir gesehen, dass darstellbare Zahlen nicht gleichverteilt auf der Zahlengerade auftreten, sondern dass der \textbf{Abstand} zwischen benachbarten darstellbaren Zahlen wächst, wenn die Zahlen grösser werden.

Mit den Fliesskommazahlen kann man auch rechnen. Wir haben insbesondere die \textbf{Addition} kennengelernt.

Wenn man zwei Fliesskommazahlen zusammen addieren möchte, muss man sie zuerst zum gleichen Exponenten bringen, dann kann man die neuen Mantissen wie ganze Zahlen addieren. Am Schluss muss man sicherstellen, dass der Exponent und die Mantisse gültig sind: der Exponent muss zwischen dem minimalen und dem maximalen Exponenten sein, die Mantisse muss eine führende Eins haben.

Wir haben gesehen, dass die Addition im Fliesskommazahlensystem Unterschiede zur Addition bei den reellen Zahlen aufweist. Erstens, nicht alle darstellbare Zahlen lassen sich exakt addieren. Zweitens, die Addition bei den Fliesskommazahlen ist \textbf{nicht assoziativ}. Es kann einen Unterschied machen, welche Teilsummen man zuerst berechnet.

Die Fliesskommazahlen sind eine mächtige Darstellung, die mit wenig Bits sehr unterschiedliche Zahlen speichern kann. Das hat aber auch seine Grenzen. Wir müssen in Kauf nehmen, dass die Resultate der Berechnungen nicht immer exakt sind.