Wir haben gesehen, wie man im Computer reelle Zahlen approximiert. Da wir eine endliche Darstellung verwenden, gibt es eine endliche Anzahl Zahlen, die wir darstellen können. Es gibt eine grösste und eine kleinste positive Zahl.

Die grösste positive Zahl kriegen wir, wenn wir in der Mantisse nur Einser schreiben und den Exponent so weit wie möglich hochdrehen. Die kleinste positive Zahl kriegen wir, wenn wir in der Mantisse eine führende Eins schreiben und sonst nur Nullen und den kleinstmöglichen Exponenten wählen.

Wir haben gesehen, dass der kleinste Abstand zwischen zwei darstellbaren Zahlen wächst, wenn die Zahlen wachsen.

Wir haben gelernt, wie man zwei Zahlen zusammen addiert. Wir haben gesehen, dass zwei darstellbare Zahlen sich nicht immer exakt addieren lassen. Wir haben auch gesehen, dass es einen Unterschied macht, in welcher Reihenfolge man Berechnungen ausführt. Die Addition ist bei Fliesskommazahlen kommutativ aber nicht assoziativ.

Die Fliesskommazahlen sind eine mächtige Darstellung, die mit wenig Bits sehr unterschiedliche Zahlen speichern kann. Das hat aber auch seine Grenzen. Wir müssen in Kauf nehmen, dass die Resultate der Berechnungen nicht immer 100\% genau sind.