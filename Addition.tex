Man kann Zahlen nicht nur speichern, sondern auch damit rechnen. In diesem Kapitel werden wir die Addition kennenlernen.

\begin{beispiel}
In diesem Beispiel rechnen wir 1/8 + 1/4.
\end{beispiel}

\begin{beispiel}
In diesem Beispiel rechnen wir 2 + 3.
\end{beispiel}

\begin{aufgabe}
Rechne folgende Summen aus. Die Mantissenlänge beträgt 5 Bits, der Exponent geht von -3 bis 3.
\begin{enumerate}[(a)]
\item 5/8 + 3/4
\item 10 + 2.25
\item 17/16 + 2
\end{enumerate}
\end{aufgabe}

Die Addition im Fliesskommazahlensystem ist wie gewöhnlich kommutativ, weil wir immer die kleinste Zahl so verschieben, dass der ''Wagen'' unter dem ''Wagen'' der grösseren Zahl liegt und dann die Bits in den ''Wagen' zusammen addieren. In der folgenden Aufgabe werden wir prüfen, ob sie auch assoziativ ist.

\paragraph{Lernaufgabe}:
Wir werden jetzt herausfinden, ob die Addition bei den Fliesskommazahlen assoziativ ist. Dazu berechnen wir zweimal die gleiche Summe.
Einmal als: 1/8 + 2/8 + 3/8 + 4/8 + 5/8 + 6/8 + 7/8 + 8/8 und einmal als: 8/8 + 7/8 + 6/8 + 5/8 + 4/8 + 3/8 + 2/8 + 1/8.
Welchen Resultat erwartest du?
Sind die zwei Summen gleich oder unterschiedlich?
Ist Addition assoziativ?
Nimm dir Zeit und rechne die zwei Summen tatsächlich aus bevor du weiter liest.

Nein, die Addition im Fliesskommazahlensystem ist nicht assoziativ, weil man nur Zahlen zusammen rechnen kann, die innerhalb der Mantissenlänge liegen.

\begin{aufgabe}\label{ein_achtel}
Betrachten wir die Summe \(1/8 + 1/8 + 1/8 + \dotsb + 1/8\).
Bei den reellen Zahlen können wir mit solchen Summen auf beliebig grossen Zahlen kommen. Bei den Fliesskommazahlen kann das nicht gehen, weil, wie wir im vorherigen Kapitel gesehen haben, es eine grösste Fliesskommazahl gibt. Aber können wir diese Zahl auch tatsächlich erreichen?

In einem Fliesskommazahlensystem mit Mantissenlänge \(5\) und Exponentenbereich von \(-3\) bis \(3\), was ist die grösste Zahl, die wir erreichen können, wenn wir beliebig viele \(1/8\) zusammen rechnen? Wie viele Summanden brauchen wir, um diese Zahl zu erreichen?
\end{aufgabe}

\subsubsection*{\textcolor{blue-violet}{Teste dich selber}}
\begin{aufgabe}\label{addition_kontrollfragen}
Beantworte folgende Fragen:
\begin{enumerate}[(a)]
\item frage grage
\end{enumerate}
\end{aufgabe}
\begin{enumerate}
\item Warum muss man den ''Wagen verschieben'' wenn man zwei unterschiedlich grosse Zahlen zusammen addiert?
\item Gregory behauptet, dass ''der Wagen bei der Summe dort stehen bleibt, wo er bei der grösseren Zahl steht''. Hat er Recht? Argumentiere.
\item Hannah behauptet, dass die Addition bei den Fliesskommazahlen nicht kommutativ und nicht assoziativ ist. Hat sie recht? Argumentiere. 
\end{enumerate}
\begin{aufgabe}\label{ameisenkönigin}
Die Ameisenkönigin möchte ausrechnen, wie viele Ameisen braucht sie, um \(10\) Reiskörnchen zu transportieren. Sie weiss, dass eine Ameise allein \(1/4\) Reiskorn transportiert. Die Ameisenkönigin hat dazu folgendes Programm geschrieben.
\begin{lstlisting}[language=Python, caption={Programm von der Ameisenkönigin}]
def nof_ameisen():
    sum = 0.0
    i = 0
    while node != 10.0:
    	i += 1
       sum += 0.25
    return i
\end{lstlisting}
Die Ameisencomputer arbeiten mit Fliesskommazahlen mit Mantissenlänge \(5\) und Exponenten zwischen \(-3\) und \(3\).
Kann die Ameisenkönigin mit diesem Programm die gewünschte Anzahl Ameisen herausfinden? Falls ja, wie viele Ameisen braucht sie, um 10 Reiskörnchen zu transportieren laut diesem Programm? Falls nein, was ist die maximale Summe, die das Programm erreichen kann?
\end{aufgabe}